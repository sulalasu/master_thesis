\minisec{Tipps: Schreiben einer wissenschaftlichen Arbeit mit \LaTeX{}}
%
Generell stellt sich für uns das Problem, nun mit \LaTeX{} einfach und rasch eine wissenschaftliche Arbeit zu schreiben.
Den Inhalt nimmt uns \LaTeX{} leider nicht ab, dafür sind wir selbst verantwortlich.
Jedoch können wir uns bei \LaTeX{} auf einige Vorteile verlassen, die wir hier näher betrachten möchten.
Vorweg sei die konsistente Formatierung genannt.

Bevor wir diese Vorteile behandeln, beschäftigen wir uns mit generellen Anforderungen an wissenschaftliche Arbeiten.
Die Erstellung der Gliederung und des Aufbaus einer wissenschaftlichen Arbeit sind meist der erste Schritt für jede Autorin bzw. jeden Autor.
Danach beschäftigt uns der Inhalt, denn jeder dieser Punkte bei der Gliederung will auch mit sinnvollem Inhalt gefüllt sein.
Ausreichendes Datenmaterial (Zahlen, Daten, Fakten) sollte dann gesammelt werden oder sein.
Gute Bilder, ansprechende Diagramme und aussagekräftige Tabellen helfen jeder Leserin und jedem Leser bei der Erfassung des wissenschaftlichen Inhalts.

\minisec{Vorgehensweise: Gliederung und Aufbau}
%
Meist ist der erste Schritt einer wissenschaftlichen Arbeit die Erstellung einer Gliederung der Arbeit (entspricht meist grob dem Inhaltsverzeichnis).
Diese wird dann mit den Betreuern der Arbeit besprochen.
Die folgende Aufzählung zeigt so eine sinnvolle Gliederung einer wissenschaftlichen Arbeit:
%
\begin{itemize}
   \item Titelblatt / Deckblatt
   \item Kurzfassung (Deutsch)
   \item Abstract (Englisch)
   \item Inhaltsverzeichnis
\end{itemize}
\begin{enumerate}
   \item Einleitung
      \begin{enumerate}
         \item Problemstellung, Motivation
         \item Vorgehensweise
      \end{enumerate}
   \item Definitionen und Abgrenzungen (Grundlagen, Theorie, Vorarbeiten)
      \begin{enumerate}
         \item Begriff A
         \item Begriff B
         \item \dots
      \end{enumerate}
   \item Hauptteil (eigene Arbeiten inkl. Ergebnisse)
      \begin{enumerate}
         \item Argument 1
         \item Argument 2
         \item \dots
      \end{enumerate}
   \item Zusammenfassung und Schlussfolgerungen (Bewertung und Ausblick)
\end{enumerate}
\begin{itemize}
   \item Literaturverzeichnis
   \item optional Abkürzungsverzeichnis, weitere Verzeichnisse und Anhang
\end{itemize}
%

\minisec{Beispiele zur Typografie}
%
Den Arbeiten \enquote{typokurz -- Einige wichtige typografische Regeln}~(\textcite{Bier:09}) sowie \citetitle{Struckmann:07} (\textcite{Struckmann:07}) entnehmen wir direkt einige Tipps:
%
\begin{enumerate}
   \item \textsc{Auszeichnungen/Hervorhebungen von Text}
      \begin{description}
         \item[Kursive] Eigene Schriftform; \emph{integrierte} 
                      Auszeichnung, die erst auffällt, wenn man an die entsprechende Stelle kommt; 
                      im Normalfall für Auszeichungen im Text am besten geeignet.
         \item[Fette] Normalerweise in Textabschnitten zu vermeiden, viel zu
                      aufdringlich (\emph{aktive} Auszeichnung), zieht direkt die Aufmerksamkeit auf sich (daher für Nachschlagewerke sinnvoll); 
                      für Überschriften, Bezeichnungen von Tabellen und Abbildungen, für Teile von Aufzählungen und Verzeichnissen sowie Tabellenköpfen geeignet;  gelegentlich wird sie auch bei Literaturverweisen im Text verwendet.
         \item[Unterstreichung] Unbedingt zu vermeiden; Überbleibsel aus dem
                      Schreibmaschinenzeitalter, als es nur eine Schriftform auf der Schreibmaschine gab.
         \item[Kapitälchen] Auch nur verwenden, wenn man weiß, was man tut.
                      Das heißt, man (er-)kennt den Unterschied zwischen echten und falschen Kapitälchen.
      \end{description}
   \item \textsc{Striche}
      \begin{description}
         \item[Trennstrich, Bindestrich] wird auch \emph{Divis} genannt und ist
                      ein kurzer Strich. Er dient zur Silbentrennung bzw. zur Verbindung zusammengesetzter Wörter. 
                      In \LaTeX{}: \verb|-|.
         \item[Gedankenstrich] Halbgeviertstrich, länger als der Divis, steht
                      zwischen zwei Leerzeichen (außer in Verbindung mit einem Satzzeichen), \zB{} Ich hoffe sehr -- und das meine ich ganz ehrlich --, Sie bald zu treffen. 
                      In \LaTeX{}: \verb|--|.
         \item[Streckenstrich/Bis-Strich] Halbgeviertstrich ohne Leerzeichen
                      davor und dahinter (Ausnahme: in Verbindung mit Wörtern wird ein Leerzeichen verwendet), \zB{} Linz--Wien, 1--2 Telefonate, 25.9.--28.12., 325 v.Chr. -- 440 n.Chr. 
                      In \LaTeX{}: \verb|--|.
         \item[Auslassungsstrich] Der Halbgeviertstrich dient im Text auch als
                      Auslassungszeichen; in Tabellen sollte dafür ein Geviertstrich (---) verwendet werden, der die Breite von zwei Nullen hat.
                      In \LaTeX{}: \verb|---|.
      \end{description}
   \item \textsc{Absatzformatierung}: Absätze kann man auf zwei Arten
            voneinander trennen: \emph{Einzug} oder \emph{Abstand}. In Bezug auf wissenschaftliche Arbeiten gilt meist: Absätze werden durch einen Einzug von ca. \SI{4}{mm} gekennzeichnet. 
            Abschnitte werden durch einen Abstand von einer Leerzeile gekennzeichnet und im Unterschied zu Absätzen ohne Einzug gesetzt.
%
      \begin{labeling}[\dots]{Flattersatz }
          \item[Flattersatz] Dieser hat den großen Vorteil, dass die 
                      Wortzwischenräume immer gleich groß sind, was positiv für die Lesbarkeit ist. 
                      Andererseits wirkt der Flattersatz eher unruhig, vor allem bei schlechtem Zeilenumbruch.
          \item[Blocksatz] Ob man sich für oder gegen Blocksatz entscheidet, ist
                      abhängig von der Zeilenlänge, der Sprache, in der der Text verfasst wird, dem Mechanismus der Silbentrennung und dem Umbruchalgorithmus der verwendeten Software.
                      Will man für längere Zeilen Blocksatz verwenden, muss man sicherstellen, dass die Software gleichmäßige und enge Wortzwischenräume erzeugt; diese sollten innerhalb einer Zeile gleich groß sein und sich von jenen in der vorangehenden und nachfolgenden Zeile nicht deutlich unterscheiden. 
      \end{labeling}
%
   \item \textsc{Schriften}: Es ist sinnvoll, für längere Texte mit breiten 
            Zeilen eine Schrift mit Serifen und Strichstärkenunterschied zu verwenden.
            Die Serifen (Endstriche) unterstützen einerseits das Auge bei der Zeilenführung und beim Zeilenrücksprung.
            Andererseits führt der Strichstärkenunterschied zu eindeutigeren Wortbildern, was das Lesen sehr erleichtert.
            Am Bildschirm sind serifenlose Schriften bzw. solche ohneStrichstärkenunterschied in der Tat häufig besser zu lesen alsserifenbehaftete Schriften. 
            Daher ist der Vergleich der Schriften auf Papier wichtig.
   \item \textsc{Trennung von Abkürzungen}: Dies ist zu vermeiden. 
            Auch abgekürzte Einheiten sollen nach Möglichkeit nicht von den dazugehörigen Zahlen getrennt werden. 
            Dazu verwenden Sie die Tilde \~{} zwischen Zahl und Einheit (noch besser: die Befehle des \texttt{siunitx}-Pakets).
\end{enumerate}

\minisec{Zitate mit einer Fußnote}
%
\Name{Christian Wolf}%
\footnote{\Name{Christian Wolf} (1679-1754), Philosoph der deutschen Aufklärung und Professor der Mathematik}
%
beschreibt 1716 in seinem mathematischen Lexikon den Ingenieur folgendermaßen:
%
\begin{quotation}
\emph{
Ingenieur, architectus militaris, ein Kriegsbaumeister, ist eine Person, welche die Kriegsbaukunst oder Fortifikation übet und also nicht allein die Festungen anzugeben vermögend ist, sondern auch die Attacken bei deren Belagerung anzuordnen weiß.}
\end{quotation}

\minisec{Formeln mit Querverweis und Kurzbefehlen}
%
Schauen wir uns einige einfache Beispiele an.
In Gleichung~\eqref{Glg:Def_a1} sehen Sie eine Formel zur Berechnung einer Krümmungszahl der Eigenwertkurven bei Stabilitätsproblemen. 
%
\begin{equation}
   a_1=-\frac{1}{2}
        \frac{\mathbf{v}_1^T 
                    \frac{\tilde{\mathbf{K}}_{T},_{\xi\xi}\lambda,_{\xi} -
                          \tilde{\mathbf{K}}_{T},_{\xi}\lambda,_{\xi\xi}}
                         {\left(\lambda,_{\xi}\right)^3}
                         \mathbf{v}_1
             }
             {\mathbf{v}_1^T 
             \frac{\tilde{\mathbf{K}}_{T},_{\xi}}{\lambda,_{\xi}} 
             \mathbf{v}_1}
      = -\frac{1}{2\,\lambda,_{\xi}}
         \left(
               \frac{\vKTxxv}     % hier verwenden wir Kurzbefehle
                    {\vKTxv}      % hier ebenso
             - \frac{\lambda,_{\xi\xi}}{\lambda,_{\xi}}
         \right)
\label{Glg:Def_a1}
\end{equation}
%

\minisec{Mehrzeilige Formeln}
%
Für diese eignet sich vor allem die \texttt{align}-Umgebung (wie in Gleichung~\eqref{Glg:align}).
%
\begin{align}
   f(x) = & f(\bar x) + \frac{(x - \bar x)}{1!} \frac{df}{dx}
   \Bigg|_{x = \bar x} +
   \frac{(x - \bar x)^2}{2!} \frac{d^2 f}{d x^2} \Bigg|_{x = \bar x} +
   \dots + \nonumber \\
   \nonumber \\
          & +\frac{(x - \bar x)^n}{n!} \frac{d^n f}{d x^n} 
            \Bigg|_{x = \bar x} +
            \frac{(x - \bar x)^{n+1}}{(n + 1)!}
            \frac{d^{(n + 1)} f}{d x^{(n + 1)}} 
            \Bigg|_{\bar x + \vartheta (x - \bar x)} \, ,
\label{Glg:align}
\end{align}
%
wobei $0 < \vartheta < 1$ ist.
%
\begin{align}
   M(x) & =  M(\bar x) &  A & = \SI{10,3}{kN}  &  M_{\max} & = \SI{85,2}{kNm}
\nonumber \\
   V(x) & = V(\bar x)  &  B & = \SI{18,7}{kN}  &  V_{\max} & = \SI{20,2}{kN}
\label{Glg:align2}
\end{align}
